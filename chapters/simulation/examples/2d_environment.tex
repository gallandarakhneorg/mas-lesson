\begin{graphicspathcontext}{{./chapters/simulation/examples/imgs/},{./chapters/simulation/examples/imgs/auto/},\old}

\begin{frame}[t]{Jaak Simulation Library}
	\begin{block}{What is Jaak?}
		\begin{itemize}
		\item \emph{Reactive agent} execution framework (but may be cognitive with additional libraries)
		\item Provides a \Emph{discrete 2D environment model}
		\item Provides a simplified agent-environment model based on LOGO-like primitives: move forward, turn left, etc.
		\end{itemize}
	\end{block}
	\begin{center}
		\includegraphics[width=.3\linewidth]{jaak_ants} \\
		\smaller\url{https://github.com/gallandarakhneorg/jaak}
	\end{center}
\end{frame}

\begin{frame}[t]{{From LOGO} to Jaak}
	\begin{block}{LOGO}
		\wrapfigure{jaak_logo_turtle}
		\begin{itemize}
		\wrapitem{Reflexive and functional programming language from the MIT}
		\wrapitem{Mainly known for its famous \Emph{graphical turtle}}
		\wrapitem{Turtles move with simple instructions: move forward, turn 45 degrees left, etc.}
		\end{itemize}
	\end{block}
	\begin{block}{Jaak}
		\wrapfigure{jaak_ants}
		\begin{itemize}
		\wrapitem{\emph{Turtles are situated agents}, which is able to move on a 2D environment}
		\wrapitem{Turtles are written in SARL}
		\wrapitem{A specific capacity provides the simple instructions for moves}
		\end{itemize}
	\end{block}
\end{frame}

%\animatedfigureslide{Architecture of a Jaak Application}{jaak_architecture}
\figureslide{Architecture of a Jaak Application}{jaak_architecture_layer1}

\begin{frame}{{Objects and Substances} in the Environment}
	\small
	\begin{block}{Environment Objects}
		\begin{itemize}
		\item Environment objects are all the entities inside the environment
		\item \alert{Turtles are not, and cannot be environmental objects}
		\item \emph{But turtle's bodies are special types of environmental objects}
		\item Each body is associated to one turtle agent, and contains the physical description of that agent: field of vision, maximal speed, weight, etc.
		\end{itemize}
	\end{block}
	\begin{block}{Substances}
		Substance is a special type of environmental object:
		\begin{itemize}
		\item It is a particular type of liquid, solid, or gas located in the environment
		\item It is countable and may be divided or expanded
		\end{itemize}
	\end{block}
\end{frame}

\figureslide[scale=.95]{{Class Diagram} of the Environment Objects}{jaakuml_Jaak_Environment_Objects}

\begin{frame}[t]{Structure of the Environment}
	\smaller
	\alertbox{The environment structure is based on a 2D grid (matrix of cells)}
	\begin{block}{Properties of the cells}
		\begin{itemize}
		\item A cell contains:
			\begin{itemize}
			\item At most one agent body (see burrow for exception), or
			\item at most one obstacle; and
			\item many other environmental objects
			\end{itemize}
		\item Cell's ``graphical'' size is application-dependent
		\end{itemize}
	\end{block}
	\vspace{-.5cm}
	\begin{columns}
		\begin{column}{.6\linewidth}
			\begin{block}{Properties of the grid}
				\begin{itemize}
				\item Size is fixed at start-up
				\item Grid could be wrapped: if an agent is located on one side of the grid and is trying to move outside; it will be moved at the opposite side of the grid
				\end{itemize}
			\end{block}
		\end{column}
		\begin{column}{.3\linewidth}
			\raisebox{-1.1\height}{\includegraphics{jaak_pacman}}
		\end{column}
	\end{columns}
\end{frame}

\figureslide[scale=.8,valign=c,label=turtlelifecycle]{{Life Cycle} of a Turtle Agent}{jaakuml_Jaak_Turtle_Life_Cycle}

\begin{frame}{{What is a} Perception?}
	\begin{definition}
		An environmental object, or a collection of environmental objects, that are inside, or intersecting, the field of perception of an agent \\
		Perception is computed and given by the sensors of the agent's body
	\end{definition}
	\begin{alertblock}{Hypotheses}
		\begin{enumerate}
		\item Agent cannot be omniscient: the scope of its perception is restricted to a limited portion of the environment
		\item Perception is for a given time
		\item Perception's list may be influenced or changed according to the properties of the sensors, e.g. visual impaired agent
		\item An agent can perceive even if it is inside a burrow
		\end{enumerate}
	\end{alertblock}
\end{frame}

\figureslide{API for the Perception}{jaakuml_Jaak_Perception_API}

\sidecite{Ferber96,Michel.07}
\begin{frame}[t]{{What is an} Influence?}
	\vspace{-.25cm}
	\smaller
	\begin{alertblock}<1->{Problem 1: simultaneous actions}
		\begin{itemize}
		\item Actions decided by different agents that may be applied at the same time on the same action space.
		\item Simultaneous actions may be under conflict. Example: when two agents try to move on the same cell.
		\end{itemize}
	\end{alertblock}
	\vspace{-.25em}
	\begin{alertblock}<2->{Problem 2: uncertain actions}
		\begin{itemize}
		\item An action decided by an agent may be skipped or partially applied according to the internal rules of the environment.
		\item Example: an agent cannot go inside the same cell of an obstacle.
		\end{itemize}
	\end{alertblock}
	\vspace{-.25em}
	\begin{block}<3>{Solution: influence}
		\begin{itemize}
		\item Actions from agents are not directly applied in the environment.
		\item Conflicts among actions are detected and solved.
		\item The resolution result is applied in the environment.
		\item \Emph{Influence: the expected action by the agent.}
		\end{itemize}
	\end{block}
\end{frame}

\figureslide{API for the Influence}{jaakuml_Jaak_Influence_API}

\end{graphicspathcontext}
