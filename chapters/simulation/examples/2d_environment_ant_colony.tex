\begin{graphicspathcontext}{{./chapters/simulation/examples/imgs/},{./chapters/simulation/examples/imgs/auto/},\old}

\sidenote{\url{https://github.com/gallandarakhneorg/jaak}}
\begin{frame}[label=antexampleslide]{{Ant Colony} Simulation}
	\begin{alertblock}{Problem}
		Simulating the global behavior of a colony of ants for finding the shortest routes from the nest to a food source.
	\end{alertblock}
	\vspace{2em}
	\begin{block}{What is a Ant?}
		\begin{itemize}
		\item Simple insects with no global knowledge and limited memory.
		\item Capable of performing simple actions:
			\begin{itemize}
			\item move
			\item get food from the cell
			\item sense pheromon is neightbor cells
			\item put pheromon into the current cell
			\end{itemize}
		\end{itemize}
	\end{block}
\end{frame}

\sidenote{\url{https://github.com/gallandarakhneorg/jaak}}
\begin{frame}<1>{Work Steps}
	\begin{center}
		\includeanimatedfigure[width=.4\linewidth]{jaak_ant_steps}
	\end{center}
\end{frame}

\sidenote{\url{https://github.com/gallandarakhneorg/jaak}}
\begin{frame}<2>[t,fragile]{Definition of the Environmental Objects}
	\vspace{-.5cm}
	\smaller
	\begin{columns}
		\begin{column}[t]{.78\linewidth}
			\begin{alertblock}{General Principle}
			For each type of object inside the environment, a subclass of \code{EnvironmentalObject} must be defined
			\end{alertblock}
			\begin{block}{In the ant colony problem}
				\begin{itemize}
				\item Nest: location where many ants may be at the same place. It is a kind of \code{Burrow} to enable this behavior
				\end{itemize}
			\end{block}
			\begin{sarllisting}[basicstyle=\scriptsize]
class Nest extends Burrow {
  val colonyId : int

  new (colonyId : int) {
   this.colonyId = colonyId
  }

  def getColonyId : int {
   this.colonyId
  }
}
			\end{sarllisting}
		\end{column}
		\begin{column}[t]{.2\linewidth}
			\raisebox{-1.1\height}{\includeanimatedfigure[width=\linewidth]{jaak_ant_steps}}
		\end{column}
	\end{columns}
\end{frame}

\sidenote{\url{https://github.com/gallandarakhneorg/jaak}}
\begin{frame}<3>[t]{Definition of the Environmental Substances}
	\smaller
	\begin{columns}
		\begin{column}[t]{.78\linewidth}
			\begin{alertblock}{General Principle}
			For each substance (countable/measurable) inside the environment, a subclass of \code{Substance} must be defined
			\end{alertblock}
			\begin{block}{In the ant colony problem}
				\begin{itemize}
				\item \code{ColonyPheromone}: put when the ant is going far and far from the ant colony
				\item \code{FoodPheromone}: put when the ant is going far and far from a food source
				\item \code{Food}: a source of food
				\end{itemize}
			\end{block}
		\end{column}
		\begin{column}[t]{.2\linewidth}
			\raisebox{-1.1\height}{\includeanimatedfigure[width=\linewidth]{jaak_ant_steps}}
		\end{column}
	\end{columns}
\end{frame}

\sidenote{\url{https://github.com/gallandarakhneorg/jaak}}
\begin{frame}<3>[t,fragile]{Definition of the Environmental Substances \insertcontinuationtext}
	\vspace{-.5cm}
	\smaller
	\begin{columns}
		\begin{column}[t]{.78\linewidth}
			\begin{block}{Definition of a food source}
				\begin{itemize}
				\item Contains the quantity of available food
				\item Quantity cannot increase
				\end{itemize}
			\end{block}
			\begin{sarllisting}[basicstyle=\scriptsize]
class Food extends FloatSubstance {
  new (foodQuantity : float) {
   super(foodQuantity, typeof(Food))
  }
  def decrement(s : Substance) : Substance {
   var oldValue = floatValue()
   decrement(s.floatValue)
   var c = clone
   c.value = abs(floatValue() - oldValue)
   return c
  }
  def increment(s : Substance) : Substance {
   // The food source could not be increased
   null
  }
}
			\end{sarllisting}
		\end{column}
		\begin{column}[t]{.2\linewidth}
			\raisebox{-1.1\height}{\includeanimatedfigure[width=\linewidth]{jaak_ant_steps}}
		\end{column}
	\end{columns}
\end{frame}

\sidenote{\url{https://github.com/gallandarakhneorg/jaak}}
\begin{frame}<4>[t]{Definition of the Environmental Processes}
	\vspace{-.5cm}
	\smaller
	\begin{columns}
		\begin{column}[t]{.78\linewidth}
			\begin{alertblock}{General Principle}
			\begin{itemize}
			\item Endogenous environmental activities are the processes that produce an evolution of the environment state outside the control of any agent
			\item Two types of endogenous environmental activities:
				\begin{enumerate}[a)]
				\item a autonomous endogenous process associated to an object
				\item the global endogenous engine
				\end{enumerate}
			\end{itemize}
			\end{alertblock}
			\begin{block}{In the ant colony problem}
				Pheromones are substances that are evaporating over the time
			\end{block}
		\end{column}
		\begin{column}[t]{.2\linewidth}
			\raisebox{-1.1\height}{\includeanimatedfigure[width=\linewidth]{jaak_ant_steps}}
		\end{column}
	\end{columns}
\end{frame}

\sidenote{\url{https://github.com/gallandarakhneorg/jaak}}
\begin{frame}<4>[t,fragile]{Definition of the Environmental Processes \insertcontinuationtext}
	\vspace{-.5cm}
	\smaller
	\begin{columns}
		\begin{column}[t]{.78\linewidth}
			\begin{block}{Update of the definition of a pheromone}
				Implements the \code{AutonomousEndogenousProcess} interface
			\end{block}
			\begin{sarllisting}[basicstyle=\scriptsize]
class Pheromone extends Substance
                implements AutonomousEndogenousProcess {
  def runAutonomousEndogenousProcess(
         currentTime : float,
         simulationStepDuration : float)
         : Influence {

   decrement(simulationStepDuration
             * this.evaporationAmount)

   if (floatValue() <= 0) {
     /* Provided by EnvironmentalObject */
     return createRemovalInfluenceForItself()
   }
   return null
  }
}
			\end{sarllisting}
		\end{column}
		\begin{column}[t]{.2\linewidth}
			\raisebox{-1.1\height}{\includeanimatedfigure[width=\linewidth]{jaak_ant_steps}}
		\end{column}
	\end{columns}
\end{frame}

\sidenote{\url{https://github.com/gallandarakhneorg/jaak}}
\begin{frame}<5>[t]{Definition of the Turtle Agents}
	\vspace{-.5cm}
	\smaller
	\begin{columns}
		\begin{column}[t]{.78\linewidth}
			\begin{alertblock}{General Principle}
			Define the different behaviors associated to the turtle agents
			\end{alertblock}
			\begin{block}{In the ant colony problem}
				\begin{itemize}
				\item \emph{Foragers}: Search for food, and carrying it to the nest
				\item \emph{Patrollers}: patrolling around the nest, and defend against others
				\end{itemize}
			\end{block}
		\end{column}
		\begin{column}[t]{.2\linewidth}
			\raisebox{-1.1\height}{\includeanimatedfigure[width=\linewidth]{jaak_ant_steps}}
		\end{column}
	\end{columns}
\end{frame}

\sidenote{\url{https://github.com/gallandarakhneorg/jaak}}
\begin{frame}<5>[t,fragile]{Definition of the Turtle Agents \insertcontinuationtext}
	\vspace{-.5cm}
	\smaller
	\begin{columns}
		\begin{column}[t]{.78\linewidth}
			\begin{block}{Definition of a patroller}
				\begin{itemize}
				\item \Emph{Define the initialization} (similar for all turtles agents)
				\item Define the specific behavior
				\end{itemize}
			\end{block}
			\begin{sarllisting}[basicstyle=\scriptsize]
agent PatrollingAnt {
  uses Lifecycle
  on Initialize {
   var body = new PhysicBodySkill
   setSkill(body)
   setSkill(new PheromoneFollowingSkill) 
   setSkill(new FoodSelectionSkill)
  }
  
  on BodyCreated {}

  on SimulationStopped {
   killMe
  }
}
			\end{sarllisting}
		\end{column}
		\begin{column}[t]{.2\linewidth}
			\raisebox{-1.1\height}{\includeanimatedfigure[width=\linewidth]{jaak_ant_steps}}
		\end{column}
	\end{columns}
\end{frame}

\sidenote{\url{https://github.com/gallandarakhneorg/jaak}}
\begin{frame}<5>[t,fragile]{Definition of the Turtle Agents \insertcontinuationtext}
	\vspace{-.5cm}
	\smaller
	\begin{columns}
		\begin{column}[t]{.78\linewidth}
			\begin{block}{Definition of a patroller}
				\begin{itemize}
				\item Define the initialization (similar for all turtles agents)
				\item \Emph{Define the specific behavior}
				\end{itemize}
			\end{block}
			\begin{sarllisting}[basicstyle=\scriptsize]
agent PratrollingAnt {
...
  uses PhysicBody

  var state = ForagerState::SEARCH_FOOD
  var bag : Food

  on Perception [state == ForagerState::SEARCH_FOOD] {
   var body = occurrence.body
		...
       moveForward(1)
       dropOff(new ColonyPheromone)
		...
   synchronizeBody
  }
}
			\end{sarllisting}
		\end{column}
		\begin{column}[t]{.2\linewidth}
			\raisebox{-1.1\height}{\includeanimatedfigure[width=\linewidth]{jaak_ant_steps}}
		\end{column}
	\end{columns}
\end{frame}

\sidenote{\url{https://github.com/gallandarakhneorg/jaak}}
\begin{frame}<6>[t]{Definition of the Turtle Spawners}
	\vspace{-.5cm}
	\smaller
	\begin{columns}
		\begin{column}[t]{.78\linewidth}
			\begin{alertblock}{General Principle}
			Two ways are available to add turtle agents in the system:
				\begin{enumerate}[a)]
				\item manual instanciation of a turtle agent through the standard SARL API
				\item definition of a spawner: a point or an area where turtle agents could be automatically created at a given generation rate
				\end{enumerate}
			\end{alertblock}
			\begin{block}{In the ant colony problem}
				A spawner is defined for each nest since the Queen is inside
			\end{block}
		\end{column}
		\begin{column}[t]{.2\linewidth}
			\raisebox{-1.1\height}{\includeanimatedfigure[width=\linewidth]{jaak_ant_steps}}
		\end{column}
	\end{columns}
\end{frame}

\sidenote{\url{https://github.com/gallandarakhneorg/jaak}}
\begin{frame}<6>[t,fragile]{Definition of the Turtle Spawners \insertcontinuationtext}
	\vspace{-.5cm}
	\smaller
	\begin{columns}
		\begin{column}[t]{.78\linewidth}
			\begin{block}{Definition of nest spawner}
				\begin{itemize}
				\item Located at a specific point on the ground
				\item Maximal number of ants to generate
				\end{itemize}
			\end{block}
			\begin{sarllisting}[basicstyle=\tiny]
class NestSpawner extends JaakPointSpawner {
  var budget : int

  new (environment : EnvironmentArea, budget : int, x : int, y : int) {
    super(environment, x, y)
    this.budget = budget
  }

  def isSpawnable(timeManager : TimeManager) : boolean {
    (this.budget > 0)
  }

  def computeSpawnedTurtleOrientation(timeManager : TimeManager) : float {
    RandomNumber::nextFloat() * 2 * Math::PI;
  }

  def turtleSpawned(turtle : UUID, body : TurtleBody,
                    timeManager : TimeManager) {
    this.budget = this.budget - 1
    body.semantic = typeof(Forager.class)
  }
}
			\end{sarllisting}
		\end{column}
		\begin{column}[t]{.2\linewidth}
			\raisebox{-1.1\height}{\includeanimatedfigure[width=\linewidth]{jaak_ant_steps}}
		\end{column}
	\end{columns}
\end{frame}

\sidenote{\url{https://github.com/gallandarakhneorg/jaak}}
\begin{frame}<7>[t]{Definition of the System Initialization}
	\vspace{-.5cm}
	\smaller
	\begin{columns}
		\begin{column}[t]{.78\linewidth}
			\begin{alertblock}{General Principle}
			Setting up the system by defining initialization functions
			\end{alertblock}
			\begin{block}{In the ant colony problem}
				\begin{itemize}
				\item Extend the Jaak kernel agent with initialization of the nests, spawners
				\item Provide the spawner-agent mapping that is used when agents are manually created
				\end{itemize}
			\end{block}
		\end{column}
		\begin{column}[t]{.2\linewidth}
			\raisebox{-1.1\height}{\includeanimatedfigure[width=\linewidth]{jaak_ant_steps}}
		\end{column}
	\end{columns}
\end{frame}

\sidenote{\url{https://github.com/gallandarakhneorg/jaak}}
\begin{frame}<7>[t,fragile]{Definition of the System Initialization \insertcontinuationtext}
	\vspace{-.5cm}
	\smaller
	\begin{columns}
		\begin{column}[t]{.78\linewidth}
			{\smaller
			\begin{block}{Initializing the system}
				\begin{itemize}
				\item \Emph{Create subtype of \code{JaakKernelAgent}}
				\item \Emph{Create the environment grid}
				\item Create the spawners on the ground
				\item Provide the spawner-agent mapping for manually created agents
				\item Initialize the UI and start the simulation
				\end{itemize}
			\end{block}}
			\begin{sarllisting}[basicstyle=\scriptsize]
agent AntColonyProblem extends JaakKernelAgent {
  def createEnvironment(tm : TimeManager) : JaakEnvironment {
    var environment = new JaakEnvironment(WIDTH, HEIGHT)
    environment.timeManager = tm
    var actionApplier = environment.actionApplier;
    for(i : 0..99)
      actionApplier.putObject(random, random,
      	new Food(50))
    return environment
  }
}
			\end{sarllisting}
		\end{column}
		\begin{column}[t]{.2\linewidth}
			\raisebox{-1.1\height}{\includeanimatedfigure[width=\linewidth]{jaak_ant_steps}}
		\end{column}
	\end{columns}
\end{frame}

\sidenote{\url{https://github.com/gallandarakhneorg/jaak}}
\begin{frame}<7>[t,fragile]{Definition of the System Initialization \insertcontinuationtext}
	\vspace{-.5cm}
	\smaller
	\begin{columns}
		\begin{column}[t]{.78\linewidth}
			{\smaller
			\begin{block}{Initializing the system}
				\begin{itemize}
				\item Create subtype of \code{JaakKernelAgent}
				\item Create the environment grid
				\item \Emph{Create the spawners on the ground}
				\item Provide the spawner-agent mapping for manually created agents
				\item Initialize the UI and start the simulation
				\end{itemize}
			\end{block}}
			\begin{sarllisting}[basicstyle=\scriptsize]
agent AntColonyProblem extends JaakKernelAgent {
...
  def createSpawners : JaakSpawner[] {
    var spawners = <JaakSpawner>newArrayOfSize(ANT_COLONY_COUNT)
    for(i : 0 .. spawners.length)
      spawners.set(i, createColony(i+1))
    return spawners
  }
  def createColony(colonyId : int) : JaakSpawner {
    ...
  }
}
			\end{sarllisting}
		\end{column}
		\begin{column}[t]{.2\linewidth}
			\raisebox{-1.1\height}{\includeanimatedfigure[width=\linewidth]{jaak_ant_steps}}
		\end{column}
	\end{columns}
\end{frame}

\sidenote{\url{https://github.com/gallandarakhneorg/jaak}}
\begin{frame}<7>[t,fragile]{Definition of the System Initialization \insertcontinuationtext}
	\vspace{-.5cm}
	\smaller
	\begin{columns}
		\begin{column}[t]{.78\linewidth}
			{\smaller
			\begin{block}{Initializing the system}
				\begin{itemize}
				\item Create subtype of \code{JaakKernelAgent}
				\item Create the environment grid
				\item Create the spawners on the ground
				\item \Emph{Provide the spawner-agent mapping for manually created agents}
				\item Initialize the UI and start the simulation
				\end{itemize}
			\end{block}}
			\begin{sarllisting}[basicstyle=\scriptsize]
agent AntColonyProblem extends JaakKernelAgent {
...
  def getSpawnableAgentType(spawner : JaakSpawner)
      : Class<? extends Agent> {
    return typeof(Ant)
  }
}
			\end{sarllisting}
		\end{column}
		\begin{column}[t]{.2\linewidth}
			\raisebox{-1.1\height}{\includeanimatedfigure[width=\linewidth]{jaak_ant_steps}}
		\end{column}
	\end{columns}
\end{frame}

\sidenote{\url{https://github.com/gallandarakhneorg/jaak}}
\begin{frame}<7>[t,fragile]{Definition of the System Initialization \insertcontinuationtext}
	\vspace{-.5cm}
	\smaller
	\begin{columns}
		\begin{column}[t]{.78\linewidth}
			{\smaller
			\begin{block}{Initializing the system}
				\begin{itemize}
				\item Create subtype of \code{JaakKernelAgent}
				\item Create the environment grid
				\item Create the spawners on the ground
				\item Provide the spawner-agent mapping for manually created agents
				\item \Emph{Initialize the UI and start the simulation}
				\end{itemize}
			\end{block}}
			\begin{sarllisting}[basicstyle=\scriptsize]
agent AntColonyProblem extends JaakKernelAgent {
...
  on Initialize {
    super._handle_Initialize_0(occurrence)
    var ui = new UI(controller)
    addJaakListener(ui)
    fireEnvironmentChange
    ui.visible = true

    controller.startSimulation
  }
}
			\end{sarllisting}
		\end{column}
		\begin{column}[t]{.2\linewidth}
			\raisebox{-1.1\height}{\includeanimatedfigure[width=\linewidth]{jaak_ant_steps}}
		\end{column}
	\end{columns}
\end{frame}

\end{graphicspathcontext}
