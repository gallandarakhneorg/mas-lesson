\begin{graphicspathcontext}{{./chapters/mas/imgs/},{./chapters/mas/imgs/auto/},\old}

\begin{frame}[t]{Interests for Using Multiagent Systems}
	\vspace{-.51cm}
	\begin{columns}
		\begin{column}[t]{.2\linewidth}
			\begin{bottomarrowsequence}
				\only<1>{\arrow[bg=CIADgreen]{Natural}}
				\only<2->{\arrow{Natural}}
				\only<2>{\arrow[bg=CIADgreen]{Distribution}}
				\only<1,3->{\arrow{Distribution}}
				\only<3>{\arrow[bg=CIADgreen]{Heterogeneity}}
				\only<1-2,4->{\arrow{Heterogeneity}}
				\only<4>{\arrow[bg=CIADgreen]{Openness}}
				\only<1-3,5->{\arrow{Openness}}
				\only<5>{\arrow[bg=CIADgreen]{Complexity}}
				\only<1-4,6->{\arrow{Complexity}}
				\only<6>{\arrow[bg=CIADgreen]{Legacy}}
				\only<1-5>{\arrow{Legacy}}
			\end{bottomarrowsequence}
		\end{column}
		\begin{column}[t]{.8\linewidth}
			\only<1>{
				\simplebox{``For many problems, the agent metaphor provides an \emph{intuitive and natural way of thinking} about, designing and implementing complex software systems'' \cite{Wooldridge09}}
				\begin{columns}
					\begin{column}{.5\linewidth}
						\begin{block}{Why?}
							\smaller
							\begin{itemize}
							\item Many real-world problems already involve
							\emph{independent, interacting entities}
							\item Agents \emph{mirror} real-world actors
							\item Decomposing a problem into agents
							\emph{matches the domain structure}
							directly, reducing the abstraction gap
							\item Stakeholders can reason using familiar concepts
							\end{itemize}
						\end{block}
					\end{column}
					\begin{column}{.5\linewidth}
						\viconbox{
							\smaller
							\begin{description}
								\item[Easier modelling] one-to-one mapping
								\item[Maintainability] changes translate naturally
								\item[Communicability] stakeholders
								understand
								\item[Scalability] new real-world
								actors $\Rightarrow$ new agents
							\end{description}
						}{pros-icon}
					\end{column}
				\end{columns}
			}
			\only<2>{
				\simplebox{``Some problems are most naturally seen as a collection of independent, autonomous components that must \emph{coordinate} to achieve a goal, where there is \emph{no single locus of control} and data may be \emph{inherently distributed}'' \cite{Wooldridge09}}
				\begin{columns}
					\begin{column}{.5\linewidth}
						\begin{block}{Why?}
							\smaller
							\begin{compactitemize}
								\item Many real systems have \emph{no natural central authority}
								\item Data is often \emph{physically or legally impossible to centralize}
								\item Single controller $=$ failure or performance bottleneck
								\item Reducing latency with \emph{local and autonomous} actions
								\item Coordination emerges local interactions
							\end{compactitemize}
						\end{block}
					\end{column}
					\begin{column}{.5\linewidth}
						\viconbox{
							\smaller
							\begin{compactdescription}
								\item[Fault tolerance] other agents continue operating
								\item[Scalability] new agent $\Rightarrow$ no redesigning
								\item[Privacy preservation] only local data
								\item[Parallelism] agents act simultaneously
								\item[Responsiveness] local decisions reduce response overhead
							\end{compactdescription}
						}{pros-icon}
					\end{column}
				\end{columns}
			}
			\only<3>{
				\simplebox{``Multiagent systems are a natural paradigm for building systems that integrate \emph{heterogeneous} components, possibly developed independently, using different languages, architectures, or data representations, [...]'' \cite{Wooldridge09}}
				\begin{columns}
					\begin{column}{.5\linewidth}
						\begin{block}{Why?}
							\smaller
							\begin{compactitemize}
								\item Systems with \emph{independently developed} components
								\item Agents have \emph{different architectures}
								\item Agents pursue \emph{different objectives}
								\item Data in \emph{different formats, ontologies, or protocols}
								\item \emph{No single model} fits all sub-problems
								\item Agents implemented with \emph{different technologies}
							\end{compactitemize}
						\end{block}
					\end{column}
					\begin{column}{.5\linewidth}
						\viconbox{
							\smaller
							\begin{compactdescription}
								\item[Specialisation] optimised agent for specific task
								\item[Emergent capabilities] combining agents with complementary behaviours
								\item[Standardised interaction] common communication protocols allow heterogeneous agents to coordinate transparently
							\end{compactdescription}
						}{pros-icon}
					\end{column}
				\end{columns}
			}
			\only<4>{
				\simplebox{``In an \emph{open} multiagent system, agents can freely enter and leave the system at runtime. The system does not assume a \emph{fixed, pre-known} set of participants, unlike \emph{closed} systems where all components are determined at design time'' \cite{Wooldridge09}}
				\begin{columns}
					\begin{column}{.5\linewidth}
						\begin{block}{Why?}
							\smaller
							\begin{compactitemize}
								\item Many systems are \emph{inherently dynamic}
								\item \emph{Impossible to enumerate} all future participants
								\item Closed systems \emph{break} when adding or removing component
								\item Open systems reflect the \emph{social and organisational} nature of real-world interactions
								\item Openness enables \emph{incremental deployment}
							\end{compactitemize}
						\end{block}
					\end{column}
					\begin{column}{.5\linewidth}
						\viconbox{
							\smaller
							\begin{compactdescription}
								\item[Dynamic scalability] no system restart
								\item[Self-organisation] system adapts its structure
								\item[Evolvability] system can incorporate new capabilities
								\item[Resilience] departing or failing agents do not halt the system
							\end{compactdescription}
						}{pros-icon}
					\end{column}
				\end{columns}
			}
			\only<5>{
				\simplebox{``Agent-oriented approaches provide an \emph{adequate paradigm} for the modelling of complex systems, since a complex system can be naturally described as a \emph{society of interacting autonomous agents}, exhibiting emergent behaviour, self-organisation, and decentralised control at multiple levels of abstraction.'' \cite{CossentinoGaudHilaireGallandKoukam2010_1}}
				\begin{columns}
					\begin{column}{.5\linewidth}
						\begin{block}{Why?}
							\smaller
							\begin{compactitemize}
								\item Complex systems consist of many
								\emph{interacting autonomous entities}
								whose global behaviour \emph{emerges}
								from local interactions
								\item \emph{No central controller}
								\item \emph{Autonomy,
								reactivity, proactivity, social ability} of complex entities \cite{Wooldridge95}
								\item \emph{Multilevel structure} of complex systems \cite{Holland.95}
							\end{compactitemize}
						\end{block}
					\end{column}
					\begin{column}{.5\linewidth}
						\viconbox{
							\smaller\smaller
							\begin{compactdescription}
								\item[Emergence] system behaviour from nonlinear interactions
								\item[Self-organisation] agents dynamically restructure their interactions
								\item[Scalability] Huge number of entities
								\item[Hierarchy] systems decomposed into complex systems
								\item[Decentralised control] complex systems are intrinsically parallel
							\end{compactdescription}
						}{pros-icon}
					\end{column}
				\end{columns}
			}
			\only<6>{
				\simplebox{\smaller ``Perhaps the most persuasive argument for agent-based computing is the \emph{legacy system} argument. Many organisations have a \emph{large investment} in software that cannot simply be discarded. The agent metaphor provides a natural way of \textbf{wrapping} these legacy components so that they can interact with newer systems'' \cite{Wooldridge09}}
				\begin{columns}
					\begin{column}{.5\linewidth}
						\begin{block}{Why?}
							\smaller
							\begin{compactitemize}
								\item Organisations accumulate \emph{decades
								of software investment} with \emph{critical business logic}
								\item Connecting legacy systems to modern architectures creates \emph{tight coupling}
								\item \emph{Interoperability} between heterogeneous legacy systems and modern services is a fundamental enterprise challenge
							\end{compactitemize}
						\end{block}
					\end{column}
					\begin{column}{.5\linewidth}
						\viconbox{
							\smaller
							\begin{compactdescription}
								\item[Agent wrapping] legacy system encapsulated behind agent
								\item[Incremental migration] from legacy system to agents progressively
								\item[Reusability] legacy system becomes reusable service to agents
							\end{compactdescription}
						}{pros-icon}
					\end{column}
				\end{columns}
			}
  		\end{column}
	\end{columns}
\end{frame}

\end{graphicspathcontext}

\endinput

