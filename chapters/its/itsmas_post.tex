\begin{graphicspathcontext}{{./chapters/its/imgs/},{./chapters/its/imgs/auto/},\old}

\subsection{ITS and AI}

\sidecite{Dale2019}
\begin{frame}[t]{{Several Usages} of AI in ITS}
	\begin{smaller}
	\begin{columns}
		\begin{column}[t]{.25\linewidth}
			\begin{tcolorbox}[title={\centering Traffic Management},height=6cm]
				\raggedright Using AI \emph{incident detection} and \emph{data collection} versus conventional algorithms
			\end{tcolorbox}
		\end{column}
		\begin{column}[t]{.25\linewidth}
			\begin{tcolorbox}[title={\centering Sensor Detection},height=6cm]
				\raggedright\emph{Datasets} that offer continuous video and sensor reading data
			\end{tcolorbox}
		\end{column}
		\begin{column}[t]{.25\linewidth}
			\begin{tcolorbox}[title={\centering Pedestrian Accessibility},height=6cm]
				\raggedright Exploring AI \emph{machine learning} to transform independent mobility for people with disabilities
			\end{tcolorbox}
		\end{column}
		\begin{column}[t]{.25\linewidth}
			\begin{tcolorbox}[title={\centering Automation},height=6cm]
				\raggedright\emph{Equip vehicles and infrastructure} with the ability to enable the safer and more efficient movement of goods and services
			\end{tcolorbox}
		\end{column}
	\end{columns}
	\end{smaller}
\end{frame}

\sidecite{Dale2019}
\begin{frame}[t]{{Cooperative Automotion} Use Cases}
	\vspace{-.5cm}
	\begin{tiny}
	\begin{columns}
		\begin{column}[t]{.25\linewidth}
			\begin{tcolorbox}[title={\larger\centering 1\\Basic Level},height=7.5cm,left=2pt,right=2pt,top=0pt,bottom=0pt]
				\raggedright
				\begin{compactitemize}
					\item Engage in a platoon defined by a geofence
					\item Leader maintains safe time gap
					\item Followers maintain interplatoon time gap
					\item Platoon size in one lane reduced from 5 to 2 cars
					\item Possible maneuvers with other cooperative ADS-equipped vehicles
				\end{compactitemize}
				\mbox{}\\\vspace{1.04cm}\includegraphics[width=\linewidth]{usecase1}
			\end{tcolorbox}
		\end{column}
		\begin{column}[t]{.25\linewidth}
			\begin{tcolorbox}[title={\larger\centering 2\\Traffic Incident Management},height=7.5cm,left=2pt,right=2pt,top=0pt,bottom=0pt]
				\raggedright
				\begin{compactitemize}
					\item Reduced command speed entering traffic incident event
					\item Determined by infield geofence
					\item Followers maintain interplatoon time gap
					\item Lane change to provide space for first responders
					\item Possible maneuvers with other cooperative ADS-equipped vehicles
				\end{compactitemize}
				\mbox{}\\\vspace{.4cm}\includegraphics[width=\linewidth]{usecase2}
			\end{tcolorbox}
		\end{column}
		\begin{column}[t]{.25\linewidth}
			\begin{tcolorbox}[title={\larger\centering 3\\Weather},height=7.5cm,left=2pt,right=2pt,top=0pt,bottom=0pt]
				\raggedright
				\begin{compactitemize}
					\item Reduced command speed entering an area with low visibility
					\item Defined by a dynamic geofence
					\item Engage in larger time gap
					\item Maintain lane guidance
					\item Possible maneuvers with other cooperative ADS-equipped vehicles
				\end{compactitemize}
				\mbox{}\\\vspace{1cm}\includegraphics[width=\linewidth]{usecase3}
			\end{tcolorbox}
		\end{column}
		\begin{column}[t]{.25\linewidth}
			\begin{tcolorbox}[title={\larger\centering 4\\Work Zones},height=7.5cm,left=2pt,right=2pt,top=0pt,bottom=0pt]
				\raggedright
				\begin{compactitemize}
					\item Reduced command speed entering work zone
					\item Defined by a stationary geofence
					\item Lane change assignment prior to entering work zone
					\item Maintain safe time gap thought the work zone
					\item Possible maneuvers with other cooperative ADS-equipped vehicles
				\end{compactitemize}
				\mbox{}\\\vspace{.8cm}\includegraphics[width=\linewidth]{usecase4}
			\end{tcolorbox}
		\end{column}
	\end{columns}
	\end{tiny}
\end{frame}

\begin{frame}{Complex System $\leftrightsquigarrow$ Multiagent System}
	\begin{alertblock}{\cite{Henderson05}}
		Artifical Intelligence, and specifically \emph{Multiagent Systems} (MAS) are considered as adapted for modeling complex systems
	\end{alertblock}
	\vfill
	\begin{block}{Multiagent systems are well suited for:}
		\begin{itemize}
			\item managing the heterogeneous nature of the system components
			\item modeling the interactions between these components
			\item trying to understand the emergent phenomena that result from these interactions
		\end{itemize}
	\end{block}
\end{frame}


\section{Conclusion on ITS}

\begin{frame}{Benefits of ITS}
	\begin{itemize}
	\item Time savings
	\vspace{1em}
	\item Better emergency response times and services
	\vspace{1em}
	\item Reduced crashes and fatalities
	\vspace{1em}
	\item Cost avoidance
	\vspace{1em}
	\item Increased customer satisfaction
	\vspace{1em}
	\item Energy and environmental benefits
	\vspace{1em}
	\item Decreasing of probability of congestion occurrence
	\end{itemize}
\end{frame}

\begin{frame}{Three Key Benefits (1/3)}
	\begin{block}{Safety}
		\begin{itemize}
		\item \emph{Road crashes cause suffering and loss of life. Many collisions occur due to the stop-start nature of traffic in congested areas} \\
			$\Rightarrow$ ITS for smoothing traffic flows, reducing congestion and reducing certain types of accidents
		\vspace{.5em}
		\item \emph{Cooperative-ITS}: involves communications between vehicles and road-side infrastructure \\
			$\Rightarrow$ improve safety by providing warnings on heavy braking or potential collisions at intersections
		\vspace{.5em}
		\item \emph{Information provided through ITS} can also be used to direct traffic away from accidents and alert emergency services as soon as the accident occurs
		\end{itemize}
	\end{block}
\end{frame}

\begin{frame}{Three Key Benefits (2/3)}
	\begin{block}{Productivity}
		\begin{itemize}
		\item Congestion:
			\begin{itemize}
			\item lowers productivity
			\item causes flow-on delays in supply-chains
			\item increases the cost of business
			\end{itemize}
		\item[$\Rightarrow$] ITS can increase productivity by finding innovative ways to increase the capacity of our current infrastructure
		\end{itemize}
	\end{block}
\end{frame}

\begin{frame}{Three Key Benefits (3/3)}
	\begin{block}{Environmental Performance}
		\begin{itemize}
		\item ITS enables the reduction of congestion and stop-start driving.
		\vspace{1em}
		\item It can also enables the reduction of fuel consumption and greenhouse gas emissions compared with normal driving conditions.
		\end{itemize}
	\end{block}
\end{frame}

\sidecite{Dale2019}
\begin{frame}{Benefits of Artificial Intelligence for ITS}
	\begin{itemize}
		\item AI enables computers to collect and analyze large amounts of data and form conclusions
		\item Improve traffic flow at intersections and specific routes
		\item Support human decision making at traffic management level
		\item Incident detection and management
		\item Traffic demand prediction
		\item Traffic signal control
		\item Real-time traffic and weather conditions
		\item Trip planning and increasing situational awareness while traveling
	\end{itemize}
\end{frame}

\begin{frame}{{Why Simulation} and Agent-Based Simulation?}
	\begin{block}{Why Simulating?}
		\begin{itemize}
		\item \emph{Too dangerous} to deploy in real World
		\item \emph{Too costly} to deploy in real World
		\item Rapid prototyping
		\item Testing standard and extrem scenarios
		\item Debugging of the algorithms
		\end{itemize}
	\end{block}
	\begin{block}{Why Agent-based Simulation?}
		\begin{itemize}
		\item Natural modeling paradigm for ITS
		\end{itemize}
	\end{block}
\end{frame}

\end{graphicspathcontext}

\endinput
